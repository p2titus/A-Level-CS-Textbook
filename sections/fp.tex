\documentclass[../alevelcs.tex]{subfiles}
\usepackage{listings}

% TODO - add code examples

\begin{document}
	In this chapter we will introduce a new paradigm of programming known as \textit{Functional Programming}. We will briefly introduce Haskell, and use it to illustrate concepts such as higher order functions, function application and partial function application. We will then move onto creating some basic functional programs.
	\tableofcontents
	\newpage
	
	\subsection{Preamble - why study Functional Programming?}
	Often, imperative programs become a mess of state and side effects. This can lead to issues when trying to use composition, as these effects can build up and become impossible to manage. Functional programming removes state from its programs entirely, and through this approach we will see that it becomes much easier to write certain classes of programs.

	Haskell is the \textit{de facto} functional programming language. As such, we will use it, but please bear in mind that it doesn't come without its downsides - in particular, its lazy evaluation\footnote{Don't worry if you don't know what this is - we will cover it later in the chapter} (amongst other aspects) makes it tricky to use in production. Despite this, it is a worthwhile language to study, as mastery will lead to thinking about programming in novel ways.

	\subsection{Function types}
	Haskell's type system is highly \textit{polymorphic} - types are very general and functions can be written very generally as a result.

	All functions have types: these describe what inputs the function accepts and the output you can expect from the function.

	Let's start with some concrete examples. The notation $:t$ represents the type of a function\footnote{We choose this notation as GHCi - the \textit{de facto} Haskell compiler-interpreter uses this macro to show types in its REPL}. These examples may seem simple, but they're designed to show how to write a function and it's accompanying \textit{type signature}.

	fact :: Int -> Int
	fact n = if n = 0 then 1 else n * fact (n-1)

	fib :: Int -> Int
	fib n = if n = 1 then 1
		else if n = 2 then 1
			else fib (n-1) + fib (n-2)
	
	addTwoNumbers :: Int -> Int -> Int
	addTwoNumbers x y = x + y

	swapTwo :: (Int, Int) -> (Int, Int)
	swapTwo (x,y) = (y,x)

	We can also write more general functions. For example, swapTwo can be written more generally as:

	swapTwo :: (a, b) -> (b, a)
	swapTwo (x,y) = (y,x)

	Nothing has changed except in the type signature! But by replacing our previously concrete type of Int with a and b, we have made our function more general! This illustrates one of Haskell's greatest strengths - \textit{type polymorphism}. This ability to write functions generally comes in useful.

	Finally, we must define some terminology. Take the following function (presented with mathematical notation rather than Haskell's notation):

	\begin{equation}
		f: A \rightarrow B
	\end{equation}

	We can see this equation \textit{maps} a value in A to a value in B. More precisely:

	\begin{itemize}
		\item A is called the \textit{domain} - the input values are all chosen from here.
		\item B is called the \textit{co-domain} - the output values are all chosen from here. Not all members of the co-domain need be in the output of the function.
		\item f is said to \textit{map} values in the domain to the co-domain.
		\item The \textit{image} of f is the set of all values that can be produced from the function.
	\end{itemize}

	\subsection{Functions as first class objects}
	Previously in imperative programs, we have seen that functions can have parameters. Consider the following fragment of pseudocode:

	% TODO - better example to motivate higher order function

			int addTwo(int x, int y) \{
				return x + y;
			\}
	
	This function takes x and y as parameters, and returns a concrete integer as a result. Whilst this allows some generality, we can write functions more generally.

	Logically, there is nothing stopping us from accepting a function itself as a parameter. In Haskell, one can write:

	
	map (+2) [1,2,3,4,5]
	
	This would produce [3,4,5,6,7]
	

\end{document}
